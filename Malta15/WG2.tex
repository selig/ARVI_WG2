\documentclass{beamer}

\usepackage{multicol}
\usepackage{color}
\newcommand{\addhere}{{\color{red} Anything Else?}}
\usepackage{url}

\usetheme{Singapore}
\AtBeginSection{\frame{\sectionpage}}

\title[WG2]{Working Group 2\\
{\bf Standardization, benchmarks, \\tool interoperability}}
\author{Chair: Giles Reger\\ Vice-chair: Eugen Zalinescu}


\begin{document}

\begin{frame}
  \titlepage
\end{frame}

\begin{frame}
  \frametitle{Outline}
  \tableofcontents  
\end{frame}

%%%%%%%%%%%%%%
\section{Aim and Objectives}
%%%%%%%%%%%%%%

\begin{frame}
\frametitle{Aim}
\Large
Clear the landscape of formalisms and tools proposed and built for runtime verification, to design common input formats and to establish a large class of benchmarks and challenges.
\end{frame}

\begin{frame}
\frametitle{Objectives}
\begin{enumerate}
	\item Enrich the RV taxonomy identified in Working Group 1 to classify tools and problems.
	\item Design common representation formats for inputs to monitoring tools organised in terms of the taxonomy.
	\item Implement interfaces, using these common representation formats, to enable different RV tools to be attached to a variety of programs and systems
	\item Create and maintain a collection of common benchmarks using the common representation formats
	\item Coordinate a regular tool competition 
\end{enumerate}
\end{frame}

\begin{frame}
\frametitle{1. Enrich the taxonomy}
\begin{itemize}
	\item The idea is to have a taxonomy of RV techniques
	\item This is a focus of WG1 but obviously effects this WG a lot
	\item[]
	\item From the perspective of this WG it is important that it allows us to separate techniques that do not easily interoperate
	\item[]
	\item The `enrichment' should involve input from the developers of tools and benchmarks so that the classification reflects reality
\end{itemize}
\end{frame}

\begin{frame}
\frametitle{2. Design common representation formats}
\begin{itemize}
	\item Will not be a one-size-fits-all approach, many aspects vary widely. For example, but not limited to, the following:
	\begin{itemize}
		\item Definition of events and their data contents
		\item Notion of time in traces
		\item Dependence between a trace and the system that generates the trace; both for online instrumentation and trace-recording
		\item Input formats for offline tools i.e. file format of traces
	\end{itemize}
	\item[]
	\item Output will be a small family of suitable languages, following the taxonomy 
	\begin{itemize}
		\item Important that taxonomy reflects the varying aspects
		\item Usability/readability concerns
		\item Use an existing language or introduce a new one?
		\item Explore translations between existing and selected languages
		\item Consider theoretic expressiveness?
	\end{itemize}
\end{itemize}
\end{frame}	

\begin{frame}
\frametitle{2. Design common representation formats (continue)}
\begin{itemize}
	\item Further considerations.... how to deal with
	\begin{itemize}
		\item Programming language specific concepts
		\item Embedded specification languages (internal DSL)
		\item Coupling with instrumentation techniques
		\item \addhere{}
	\end{itemize}
	\item[]
	\item Need to have enough formats to separate classes but not too many that there is one tool per class!
\end{itemize}
\end{frame}

\begin{frame}
\frametitle{3. Implement interfaces}
\begin{itemize}
	\item Idea is to separate monitoring systems and the systems they monitor via common interfaces
	\item[]
	\item Currently RV tools are generally either
	\begin{enumerate}
		\item offline and system agnostic... but not useful for online
		\item outline requiring events to be submitted manually
		\item inline (online) and instrumentation-technique dependent
	\end{enumerate}
	\item Observations
	\begin{itemize}
		\item If case 1 are incremental then they can be used outline
		\item Inline tools can typically be separated into instrumentation+outline
	\end{itemize}
	\item Therefore if we
	\begin{enumerate}
		\item Create an interface for outline monitoring
		\item Define one (or more) programming language specific instrumentation technique(s) to use this interface
		\item Extend monitoring techniques to instantiate this interface
	\end{enumerate}
	\item Then RV tools can be programming language agnostic
\end{itemize}
\end{frame}	
	

\begin{frame}
\frametitle{3. Implement interfaces (continued)}	
\begin{itemize}
	\item There are some issues to consider when implementing interfaces
	\begin{itemize}
		\item Interfaces should be modulo the taxonomy
		\item Some tools depend on language-dependent constructs internally. Should these must be captured by the interface?
		\begin{itemize}
			\item Identity i.e. semantic (equals) vs reference (==) in Java
			\item Garbage collection in Java
		\end{itemize}
		\item Data parameters as runtime objects i.e. passing a collection object to a monitor and then calling the \texttt{size} method from inside the monitor.
		\item \addhere{}
	\end{itemize}
\end{itemize}
\end{frame}

\begin{frame}
\frametitle{4. Collect benchmarks}
\begin{itemize}
	\item Idea: to collect benchmarks that can be used by the community to compare techniques and encourage/focus research
	\item Bonus: can be used by the competition
	\item[]
	\item The taxonomy and common representation formats would be used
	\item Need to also record meta-information i.e. description, references, comments
\end{itemize}
\end{frame}

\begin{frame}
\frametitle{4. Collect benchmarks (continued)}
\begin{itemize}
	\item The aim of this WG mentions {\bf challenges}
	\item What does it mean for a benchmark to be challenging?
	\begin{itemize}
		\item Require efficient monitors? i.e. long traces, lots of data
		\item Require responsive monitors? i.e. impose maximum per-event overhead
		\item Require expressive languages? i.e. complex concepts
		\item Require usable tools?
		\item \addhere{}
	\end{itemize}
\end{itemize}
\end{frame}

\begin{frame}
\frametitle{5. Coordinate competition}
\begin{itemize}
	\item Already in progress.
	\item But still finding its feet.
	\item Should discuss what best supports the community
	\begin{itemize}
		\item Promote further research
		\item Encourage collaboration and cooperation		
		\item Consider emerging subfields? (e.g. distributed monitoring)
		\item Consider the usability aspect?
		\item Should we evaluate tools or monitoring techniques; if the latter we should separate instrumentation and trace parsing methods
		\item \addhere{}
	\end{itemize}
\end{itemize}
\end{frame}

%%%%%%%%%%%%%%
\section{Current Status}
%%%%%%%%%%%%%%

\begin{frame}
\frametitle{CRV14}
\begin{itemize}
	\item First competition
	\item Very successful
	\item Running again this year (15)
	\item[]
	\item There is an upcoming journal paper giving lots of details
\end{itemize}
\end{frame}

\begin{frame}
\frametitle{CRV14 Benchmarks}
\begin{itemize}
	\item Domains \begin{multicols}{2}
	\begin{itemize}
		\item Banking
		\item Concurrency
		\item Aerospace
		\item Programming (Java API)
		\item Databases
		\item Networking
		\item And others...
	\end{itemize}
	\end{multicols}
	\item Languages
	\begin{itemize}
		\item Temporal logics: LTL, MFOTL, LTL + FO theories, LOLA
		\item Automata based: DATEs (enriched FSAs), QEA, prm4j
		\item Rule-based (LogFire)
		\item JavaMOP can use LTL, ERA, FSM, CFG, SRS 
	\end{itemize}	
	\item Input formats
	\begin{itemize}
		\item Traces - organisers defined CSV, XML and JSON formats
		\item Program instrumentation - AspectJ, Java reflection, manual
	\end{itemize}	

\end{itemize}
\end{frame}


\begin{frame}
\frametitle{CRV14 Tools}
\begin{itemize}
	\item  Offline monitoring
	\begin{itemize}
		\item RiTHM2
		\item Monpoly
		\item STePr
		\item MarQ
	\end{itemize}
	\item  Online monitoring of Java programs
	\begin{itemize}
		\item Larva
		\item JUnitRV-MMT
		\item JavaMOP
		\item MarQ
	\end{itemize}	
	\item  Online monitoring of C programs
	\begin{itemize}
		\item RiTHM
		\item E-ACSL
		\item RTC
	\end{itemize}	
	\item[]
	\item Others entered but did not make it to the evaluation stage (6 dropped out)
\end{itemize}
\end{frame}

\begin{frame}
\frametitle{CRV14 Benchmarks, other observations}
\begin{itemize}
	\item Lack of common representations made exporting properties difficult. 
	\begin{itemize}
		\item  It was often the case that properties were not specified equivalently in different languages
		\item i.e. usually a small set of corner cases are valid in one and invalid in another
		\item No easy way to check equivalence... does it matter?
	\end{itemize}
	\item[]
	\item Almost all benchmarks dealt with \emph{data} in some way
	\begin{itemize}
		\item Trace slicing, first-order temporal logic
		\item Treating runtime objects as data (i.e. calling methods on them)
	\end{itemize}
	\item[]
	\item Lack of clear notion of instrumentation and input formats for online monitoring meant that it was difficult to tell if the same events were being monitored
\end{itemize}
\end{frame}

\begin{frame}
\frametitle{RV tools in general}
\begin{itemize}
	\item There does not exist a recent survey of RV tools (correct if wrong), or even a list (beyond those entering the competition)
	\item[]
	\item The 2004 ``A Taxonomy and Catalog of Runtime Software-Fault Monitoring Tools'' by Delgado et al. is the only tool-focussed survey to date and is over 10 years out of date
	\item[]
	\item \addhere{}
\end{itemize}
\end{frame}

\begin{frame}
\frametitle{Benchmarking in RV Research}
\begin{itemize}
	\item Very little focus on developing benchmarks
	\item[]
	\item The DaCapo benchmarks + some Java API properties are commonly used
	\begin{itemize}
		\item These are very restricted to a single domain and quite low-level
		\item The combination of properties+programs are all quite similar in exhibited behaviour
	\end{itemize}
	\item[]
	\item The recent SMock platform is a step towards better benchmarking systems. But more can be done to reflect different domains.
	\item[]
	\item \addhere{}
\end{itemize}
\end{frame}

\begin{frame}
\frametitle{Interoperability in RV Research}
\begin{itemize}
	\item By interoperability here we mean a focus on making RV tools work together, either directly or via standardisation of inputs
	\item[]
	\item It does not seem that this has been a key focus of any research efforts (correct if wrong)
	\item[]
	\item There are some unifying factors
	\begin{itemize}
		\item Finite-trace LTL definitions are often reused
		\item Multiple systems now use the notion of trace-slicing
		\item Some instrumentation techniques (AspectJ) unify systems
		\item \addhere{}
	\end{itemize}
	\item[]
	\item \addhere{}
\end{itemize}
\end{frame}

%%%%%%%%%%%%%%
\section{Planning}
%%%%%%%%%%%%%%

\begin{frame}
\frametitle{Support taxonomy development}
\begin{itemize}
	\item Part of Working Group 1
	\item Objectives 2 and 3 depend on this
	\item[]
	\item Talk to WG1 and see how we can support this
	\item Ensure that development takes pragmatic issues into account
\end{itemize}
\end{frame}

\begin{frame}
\frametitle{Create benchmark/tool repository}
\begin{itemize}
	\item Can start without taxonomy... but will need to be able to extend to support this
	\item[]
	\item Idea: Central website that stores benchmarks and tools
	\item What is a benchmark in this respect?
	\begin{itemize}
		\item Informal and formal descriptions
		\item Traces and/or programs
		\item ? specifications in multiple RV tools
	\end{itemize}
	\item[]
	\item Collecting/submitting benchmarks should be independent of competition i.e. for the benefit of the whole community
	\item Encourage input from WG3 and WG4
	\item[]
	\item Consider adapting benchmarks from other program verification communities?
\end{itemize}
\end{frame}

\begin{frame}
\frametitle{Support competition}
\begin{itemize}
	\item Best way to support competition is to enter it
	\item[]
	\item Might still be possible to enter this year... but it is already well under way. Talk to the chairs. (This is at time of initial distribution; by the Malta meeting it is likely it will be too late)
	\item[]
	\item Consider setting up permanent website (potentially same as benchmark repository)?
	\item Discuss future iterations
\end{itemize}
\end{frame}

\begin{frame}
\frametitle{CRV15 benchmarks}
\begin{itemize}
	\item This year's competition is trying to create a repository of all artefacts. See here
	\item[] \url{https://forge.imag.fr/plugins/mediawiki/wiki/crv15/index.php/Main_Page}
	\item[]
	\item This year there will also be prizes for `best' benchmarks, which will hopefully involve the community
\end{itemize}
\end{frame}

\begin{frame}
\frametitle{Timeline and What's next}
\begin{itemize}
	\item Official Milestone(s) for WG2: 
          \begin{itemize}
          \item End of 2nd year: First version of the electronic
            infrastructure and benchmarks
          \item End of 3rd year: Second version of the infrastructure
            and benchmark, infrastructure for competitions
          \item End of 4th year: Final electronic infrastructure and
            competition.
          \end{itemize}
        \item One of the meeting's goal: define a more detailed timeline
\end{itemize}
\end{frame}

\end{document} 
